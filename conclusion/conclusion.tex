\chapter{Conclusion}
Overall, the project went successfully.
There are certain functionalities that have to remain incomplete, but they will stay for future developers to continue.
However, the majority of planned functionality was completed, as expected.\\[1pt]
The 360-degree camera was successfully integrated as part of the Unity Game Engine environment to receive and process video streams from the real world and recreate it in the virtual one.
It went through several possible implementations, but in the end, a surrounding hemisphere was able to interpret a video as part of the gaming environment.
A PIXPRO camera itself also had a second purpose, to serve as a Network point to all devices in this environment.
Due to the specific capabilities of a particular model, that idea requires a better tool.
A computer with a Game Engine server as a network point to Unite all parts.
There are specific difficulties with camera which went unattended, for example, the initial start of the recording requires actual PIXPRO app running somewhere within Network.
All similar concerns, including those already mentioned, are the issue of the design itself.\\[1pt]
An SPB device, Raspberry Pi 3 in particular, one more time has proven to be a potent tool.
Despite serving as a controller for a tank, it also helped with a few additional functionalities, which are beyond the scope of the research. For example, such an SPB device may serve as a mighty security bridge, as well as a Network Router.
The primary goal remained the same. Through simple Python language and GPIO port interaction, it turned into a reliable control tool for a toy, capable of communicating with any other environment.
For this project, communication through ZMQ messages between R-pi and Unity served well, with Python and C\# languages.
Indeed, this device can prove to be useful in all kinds of similar projects.\\[1pt]
One more time, proper equipment with proper software made the entire project life much more comfortable.
HTC Vive Gear allowed for the reconstruction of a realistic environment as well as supporting the development process itself.
SteamVR prefab contained a lot of implemented tools to support developers in this relatively new area.
All Software changes cause specific difficulties with learning and staying up to date with the latest software, but after overcoming that issue, creating an event system within a game becomes only a matter of providing correct inputs.
Maybe VR is not a very flexible environment yet, but it still has a long way to go.\\[1pt]
As a result, this project has proven that when people talk about Mixed or Augmented realities, it does not have to be some new equipment capable of modifying human perception on the run.
Such technologies indeed have proven to be useful, but they are still achievable with a more brute force approach, provided they have a solid idea and proper tools.
% A perfect point to start the presentation is that AR does not have to be aGear only. It can be a set of specific types of equipment.
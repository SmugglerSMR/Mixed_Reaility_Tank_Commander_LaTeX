%\vspace{-6pc}
\chapter{Introduction}
%\vspace{-2pc}
The final report aims to provide progress, results, and a summary of one university year of work under the Mixed Reality Project.
Since the first proposal, it went through several changes, rethinking and suffered from changes in core implementations.
It will provide the final result of the development, alongside demonstration footage, a solution to some problems and difficulties that were left unchallenged due to lack of time and scope of development.
In the end, it will provide a reflection of the entire work as well as some future goals for another researcher to take place.
Several timelines will be added to the report's end to highlight the time each implementation took.
Overall, the report will state the reasons why investigation can be considered successful and what possibilities it may open from both research and entertainment perspectives.
\section{Objectives}
The investigation aimed to demonstrate how virtual reality, combined with real-world views, may enhance human vision and produce entirely new experiences on an unprecedented scale.
A combination of 360 camera with Virtual Reality is supposed to open a new perspective on the world from both in-game perspectives and the real world.
A small 1/16-scale toy within a reconstructed testing environment is supposed to be one way to introduce practical usage.
At the same time, a set of High Computation Devices will be used to simplify the computation process to make the experience as immersive as possible.
Some low-level techniques in managing computational resources will allow to synchronise actions and make the simulation run smoother.
\section{Challenges}
The project's primary working environment ended up being the Unity Game Engine. The tool allowed comfortable integration of all parts of the project inside a single scene to create a well-defined simulation.
However, researches primary skills based around Computer and Software System Bachelor Degree, any skills within a Gaming Industry considered to be completely new, which significantly affected overall performance and overall look of the final work.
\section{Contributions}
Amongst all contributors except for the supervisor, the person who contributed significantly to the project was Rodney Zsolczay, who generously provided his Final year Bachelor's Degree Project to support Virtual Reality Research.
Provided prefab contained a completely functional Tank, with animations for movement, firing and self-destruction. 
As a result, the provided solution was spared from developing a presentable example of the simulation.
It allowed me to focus on the more relevant matters as an Engineer and leave other minor details to Game Developers. \\
The development of a machine learning (ML) model went through consultations with Professor Gonzalez about Engineering work standards and supervision for the ML model creation with PhD candidate Olga Moskvina.
Their example of animal detection assignment established a foundation for Neural Networks (NN) to recognise opponents' tanks from a video stream.
With the support of  Dr Mahsa Baktashmotlagh and Pr Jim Hogan from UQ and QUT, the project was meant to grow from local video processing to Cloud-based, using the performance of Amazon Virtual Machines (AWS) through JavaScript.
In the end, the idea was not even discussed, but it gave a functional area for research in the future.\footnote{As an example, some US companies opened cloud gaming.}